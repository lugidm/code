In dieser Arbeit wird evaluiert, wie gut sich das globale Klimamodell wie hier z.B. das MPI-ESM-LR auf regionalen Maßstab skalieren lässt. Dazu werden die Daten von zwei unterschiedlich skalierten regionale Klimamodellen mit Werten aus der Vergangenheit verglichen. Hauptsächlich soll dabei auf Starkregen-Ereignisse eingegangen werden. Evaluiert wurden dabei ALP-3 und EUR-11. ALP-3 sind  die Ergebnisse einer Konvektions-erlaubenden Klimasimulation mit dem regionalen Klimamodell CCLM5-0-9, wobei dafür EUR-11 mittels CCLM5-0-9 auf eine Auflösung von 3 km gebracht wurde. EUR-11 sind die Ergebnisse der angetrieben mit CCLM4-8-17, einem Standard-Klimamodell, mit 12.5km Auflösung - damit sind regionale Konvektionszellen nicht simulierbar, da diese sich auf wesentlich kleineren Maßstäben abspielen.\\
Beide regionalen Klimamodelle (ALP-3 und EUR-11) wurden jeweils mit drei unterschiedlichen Datensätzen angetrieben: 
\begin{itemize}
	\item mit Re-Analysedaten der Periode 1996-2005
	\item mit historischen Daten der Klimasimulation MPI-ESM-LR (r2i1p1) aus der Periode 1995-2005
	\item mit Zukunftsdaten der Klimasimulation MPI-ESM-LR und dem Treibhausszenario RCP8.5, dem höchsten der RCP-Szenarios, mit Daten aus der Periode 2090-2099
\end{itemize}
Die Daten werden dabei auf unterschiedliche Arten mit dem Beobachtungs-Datensatz verglichen. Die Beobachtungsdaten (Im Folgenden auch als observation-Daten bezeichnet) kommen aus dem Datensatz (Im Folgenden auch als apgd-Daten bezeichnet) kommen aus dem EURO4M APGD Projekt unter Führung der MeteoSwiss - Behörde\cite{meteoswiss}. Dieser Datensatz hat eine Auflösung von 5x5km, beinhaltet aber keine Temperaturdimension, diese wird über ein ''remaping'' der E-OBS Temperatur-Daten berechnet.
Da viele Arten der Validierung von Klimamodellen in Fachkreisen besprochen werden, fokussiere ich mich hauptsächlich auf die in ''VALUE: A framework to validate downscaling approaches for climate change studies'' \cite{maraun_value} beschriebenen Herangehensweisen, werde mich jedoch nicht vollends auf diese beschränken.