In dieser Arbeit werden die Simulationsdaten aus zwei unterschiedlichen regionalen Klimamodellen, evaluiert und verglichen. Der hauptsächliche Unterschied liegt in der Repräsentation der Konvektion: parametrisiert oder dynamisch modelliert. Auf die genauen Unterschiede der beiden Modellen wird in Kapitel \ref{chap:modells} eingegangen.\\
Um Daten aus einem globalen Klimamodell auf regionale Ebene zu bringen, reichen statistische Methoden kaum aus (vgl. D.Maraun und Statistical Downscaling and Bias Correction for Climate Research \cite{statistical_downscaling}). Es muss ein regionales Klimamodell angewendet werden, um die Daten des grob-aufgelösten globalen Klimamodells auf die lokalen Gegebenheiten korrekt projizieren zu können. Die in dieser Arbeit evaluierten regionalen Klimamodelle sind:
\begin{itemize}
	\item CCLM5-0-9: Ein Klimamodell mit einer Auflösung von 3km, welches Konvektion simuliert. Die Ergebnisdaten werden als ALP-3 bezeichnet.
	\item CCLM4-8-17: Ein Standard-Klimamodell mit einer Auflösung von 12.5km, damit sind regionale Konvektionszellen nicht simulierbar, da sich diese auf wesentlich kleineren Maßstäben abspielen. Deshalb und aufgrund von beschränkter Rechenleistung wurden die Konvektion in diesem Modell parametrisiert. Die daraus gewonnen Simulationsdaten werden als EUR-11 bezeichnet.
\end{itemize}
Beide regionalen Klimamodelle (ALP-3 und EUR-11) wurden jeweils mit drei unterschiedlichen Daten angetrieben: 
\begin{itemize}
	\item Re-Analysedaten der Periode 1996-2005
	\item historischen Daten der Klimasimulation MPI-ESM-LR (r2i1p1) \cite{mpi-esm-lr} aus der Periode 1995-2005
	\item Zukunftsdaten (2090-2099) des Klimamodells MPI-ESM-LR \cite{mpi-esm-lr} für das Treibhausszenario RCP8.5, dem höchsten der RCP-Szenarios.
\end{itemize}
Zur Evaluierung werden die Daten dabei auf unterschiedliche Arten mit dem Beobachtungs-Datensatz verglichen. Die Beobachtungsdaten kommen aus dem Datensatz (im Folgenden auch als APGD-Daten bezeichnet) des EURO4M APGD Projekt unter Führung der MeteoSwiss - Behörde\cite{meteoswiss}.(Für nähere Informationen zum Datensatz siehe:\cite{apgd}). Dieser Datensatz hat eine Auflösung von 5x5km, beinhaltet aber keine Temperaturdimension, diese wird über ein ''remaping'' der Temperatur-Daten im E-OBS Datensatz berechnet, ein Datensatz mit einer Auflösung von $0.25\deg$. Entstanden ist dieser Datensatz als ein europäisches Ensemble an Beobachtungsdaten nationaler Behörden: Für nähere Informationen siehe \cite{eobs}.
Da viele Arten der Validierung von Klimamodellen in Fachkreisen besprochen werden, fokussiere ich mich hauptsächlich auf die in ''VALUE: A framework to validate downscaling approaches for climate change studies'' \cite{maraun_value} beschriebenen Herangehensweisen, werde mich jedoch nicht vollends auf diese beschränken.