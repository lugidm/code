In dieser Arbeit wird evaluiert, wie gut sich das globale Klimamodell MPI-ESM-LR \cite{mpi-esm-lr} auf regionalen Maßstab herunterbrechen lässt. 
Um Daten aus einem globalen Klimamodell auf regionale Ebene zu bringen reichen statistische Methoden kaum aus, es muss ein regionales Klimamodell kreiert werden um diese Daten auf die lokalen Gegebenheiten zu projizieren. Die in dieser Arbeit evaluierten regionalen Klimamodelle: ALP-3 und EUR-11. ALP-3 sind  die Ergebnisse einer Konvektions-erlaubenden Klimasimulation mit dem regionalen Klimamodell CCLM5-0-9, wobei dafür EUR-11 mittels CCLM5-0-9 auf eine Auflösung von 3 km gebracht wurde. EUR-11 sind die Ergebnisdaten des Modells CCLM4-8-17, einem Standard-Klimamodell, mit 12.5km Auflösung - damit sind regionale Konvektionszellen nicht simulierbar, da diese sich auf wesentlich kleineren Maßstäben abspielen. Deshalb und aufgrund von beschränkter Rechenleistung wurden die Konvektion in diesem Modell parametrisiert\\
Beide regionalen Klimamodelle (ALP-3 und EUR-11) wurden jeweils mit drei unterschiedlichen Daten angetrieben: 
\begin{itemize}
	\item mit Re-Analysedaten der Periode 1996-2005
	\item mit historischen Daten der Klimasimulation MPI-ESM-LR (r2i1p1) aus der Periode 1995-2005
	\item mit Zukunftsdaten der Klimasimulation MPI-ESM-LR und dem Treibhausszenario RCP8.5, dem höchsten der RCP-Szenarios, mit Daten aus der Periode 2090-2099
\end{itemize}
Zur Evaluierung werden die Ergebnisdaten dabei auf unterschiedliche Arten mit dem Beobachtungs-Datensatz verglichen. Die Beobachtungsdaten kommen aus dem Datensatz (im Folgenden auch als apgd-Daten bezeichnet) des EURO4M APGD Projekt unter Führung der MeteoSwiss - Behörde\cite{meteoswiss}.(Für nähere Informationen zum Datensatz siehe:\cite{apgd}). Dieser Datensatz hat eine Auflösung von 5x5km, beinhaltet aber keine Temperaturdimension, diese wird über ein ''remaping'' der E-OBS Temperatur-Daten berechnet, ein Datensatz mit einer Auflösung von $0.25\deg$. Entstanden ist dieser Datensatz als ein europäisches Ensemble an Beobachtungsdaten nationaler Behörden: Für nähere Informationen siehe \cite{eobs}.
Da viele Arten der Validierung von Klimamodellen in Fachkreisen besprochen werden, fokussiere ich mich hauptsächlich auf die in ''VALUE: A framework to validate downscaling approaches for climate change studies'' \cite{maraun_value} beschriebenen Herangehensweisen, werde mich jedoch nicht vollends auf diese beschränken.