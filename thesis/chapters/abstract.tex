In dieser Arbeit soll evaluiert werden wie gut sich das globale Klimamodell EUR-11 auf regionalen Maßstab skalieren lässt. Dazu werden die Daten von zwei unterschiedlich skalierten regionale Klimamodellen mit Werten aus der Vergangenheit verglichen. Hauptsächlich soll dabei auf Starkregen-Ereignisse wie zum Beispiel Gewitterzellen eingegangen werden. Evaluiert werden dabei ALP-3 und EUR-11. ALP-3 sind  die Ergebnisse einer \glqq konvektions-erlaubenden\grqq{} Klimasimulation mit dem regionalen Klimamodell CCLM5-0-9, wobei dafür EUR-11 mittels CCLM5-0-9 auf eine Auflösung von 3 km gebracht wurde.