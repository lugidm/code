In dieser Arbeit wird evaluiert, wie gut sich globale Klimamodelle wie z.B. MPI-ESM-LR auf regionalen Maßstab skalieren lässt. Dazu werden die Daten von zwei unterschiedlich skalierten regionale Klimamodellen mit Werten aus der Vergangenheit verglichen. Hauptsächlich soll dabei auf Starkregen-Ereignisse wie zum Beispiel Gewitterzellen eingegangen werden. Evaluiert wurden dabei ALP-3 und EUR-11. ALP-3 sind  die Ergebnisse einer Konvektions-erlaubenden Klimasimulation mit dem regionalen Klimamodell CCLM5-0-9, wobei dafür EUR-11 mittels CCLM5-0-9 auf eine Auflösung von 3 km gebracht wurde. EUR-11 sind die Ergebnisse der angetrieben mit CCLM4-8-17, einem Standard-Klimamodell, mit 12.5km Auflösung - damit sind regionale Konvektionszellen nicht simulierbar, da diese sich auf wesentlich kleineren Maßstäben abspielen.\\
Beide regionalen Klimamodelle (ALP-3 und EUR-11) wurden jeweils mit drei unterschiedlichen Datensätzen angetrieben: 
\begin{itemize}
	\item mit Re-Analysedaten der Periode 1996-2005
	\item mit historischen Daten der Klimasimulation MPI-ESM-LR (r2i1p1) aus der Periode 1986-2005
	\item mit Zukunftsdaten der Klimasimulation MPI-ESM-LR und dem Treibhausszenario RCP8.5, dem höchsten der RCP-Szenarios, mit Daten aus der Periode 2090-2099
\end{itemize}

The climate system is complex and high-dimensional, and models are not intended to be isomorphisms of nature [Stainforth et al., 2007]; thus no climate model or downscaling method can be expected to
reproduce all aspects of the system perfectly, and a validation of all aspects would be practically impossible. However, in any given application only a small part of the system will be relevant: specific variables or phenomena, at specific space and time scales in a specific region. A user focused approach to validation must therefore start by identifying the phenomena and scales of interest; with respect to these, it must seek to identify the key strengths and weaknesses of a method. For a given application, it has to give advice whether a method performs well or even better than other methods, and where it is likely to fail. In this way, users can determine whether a particular method is appropriate for their application, and
can compare methods.