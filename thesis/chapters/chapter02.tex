Für eine eventuelle BIAS-Korrektur kommt eine bisherige Herangehensweise nicht in Frage, da dadurch die lokale und vor allem zeitliche Varianz zu stark beeinträchtigt wird wie in Abb. \ref{fig:boxplot_corrected_rcp} zu erkennen ist. Laut D.Maraun \cite{biasMaraun} ist dafür ein Herangehensweise über stochastische Quantile-Mapping nötig. Dabei ist vor allem auf Überkorrektur acht zu geben, welche sich durch die Überschätzung der Flächenmittelwerte für Extremniederschläge und falsche Trends äußert. Die von Maraun in \cite{biasMaraun} postulierte Herangehensweise wird in diesem Kapitel verfolgt.
\section{Herangehensweise}
