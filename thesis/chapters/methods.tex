Wie im Abstrakt bekanntgegeben, werden in dieser Arbeit verschieden Methoden zur Evaluierung angewendet. Durch diese mehrfachen Vergleiche können unterschiedliche Wetterphänomene nachgewiesen werden.\\

\begin{table}[h]
	\begin{tabularx}{\textwidth}{|X|X|X|}
		\hline
		\textbf{Methode} & \textbf{nachgewiesenes Phänomen}& \textbf{Einschränkungen}\\
		\hline
		Jahresmittel [Kap. \ref{chap:mean}] & Übereinstimmung der geographischen Regenzonen & Starkregenereignisse werden nicht abgebildet.\\
		\hline
	\end{tabularx}
\caption{Verwendete Evaluierungsmethoden}
\end{table}
\hfill\\
Um die Beobachtungsdaten aus dem APGD\cite{meteoswiss} Datensatz mit den simulierten Daten (EUR-11) vergleichen zu können wurden die Beobachtungsdaten mittels cdo (für weitere Informationen siehe \cite{cdo}) auf dasselbe Raster wie die Simulationsdaten gebracht. Dies geschah über eine Interpolation: hierbei wurde die Funktion ''ramapbil'' von cdo verwendet, die eine bilineare Interpolation des gröberen Datengitters auf das feiner vornahm. Das heist, dass zunächst eine lineare Interpolation in die eine und dann in die andere Richtung vorgenommen wurde, dabei wurde der Abstand einer feineren Gitterzelle zu den Mittelpunkten der umgebenden gröberen Gitterzelle als Gewicht der Interpolation genommen. So wurde ein feineres über den Abstand gemitteltes Gitter aus dem Gröberem errechnet.\\
Um die Daten des simulierten ALP-3 Datensatzes mit den Beobachtungsdaten (APGD \cite{meteoswiss}) vergleichen zu können wurden die Simulationsdaten auf ein gröberes 5x5km Gitter aggregiert: dies geschah über die Funktion remapcon von cdo., was einem konservativem remappen entspricht. 
