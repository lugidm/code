Wie im Kapitel \ref{chap:quantile} gezeigt wurde scheint sich die Abweichung des 99. Quantils des Niederschlags (und damit des Starkregens) bei allen Datensätzen nahezu identisch abzuzeichnen. Deshalb soll in diesem Kapitel wie es in \cite{biasMaraun} gemacht wurde, gezielt auf die Starkregenereignisse der unterschiedlichen Jahreszeiten eingegangen werden um die Evaluation der beiden Klimamodelle abzuschließen.
\section{Herangehensweise}
\begin{itemize}
\item Alle Datensätze werden auf die Jahreszeiten aufgeteilt:
	\subitem Frühling: 20.März - 20.Juni
	\subitem Sommer: 20.Juni - 22. September
	\subitem Herbst: 22.September - 21. Dezember
	\subitem Winter: 21. Dezember - 20. März
\item Von diesen neu aufgeteilten Datensätzen werden die 99. Quantile berechnet
\item Diese Quantile werden mit den Beobachtungsdaten verglichen.
\end{itemize}

