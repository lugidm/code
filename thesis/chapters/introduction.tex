Das downscaling von Global Climate Models (GCM) ist ein hochkomplexes Thema der Klimaforschung. Ein GCM liefert im generellen recht gute Aussagen über die Entwicklung des globalen Klimas. Diese Ozean-Atmosphären gekoppelten Zirkulationsmodelle beruhen meistens auf einer parametrisierung der Konvektion, da Konvektion im relativ groben Raster eines GCM nicht simuliert werden kann. Aus dieser Parametrisierung entstehen grobe Fehler in den GCM's. \cite[see][Stevens \& Bony]{stevensbony}
Unsere Atmosphäre von vielen mikrophysikalischen Prozesse beeinflusst, die noch nicht alle zur Gänze erforscht sind