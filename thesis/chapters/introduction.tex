Das downscaling von Global Climate Models (GCM) ist ein hochkomplexes Thema der Klimaforschung. Ein GCM liefert im generellen recht gute Aussagen über die Entwicklung des globalen Klimas, jedoch beruhen diese Ozean-Atmosphären gekoppelten Zirkulationsmodelle aber meistens auf einer Parametrisierung der Konvektion, da diese im relativ groben Raster eines GCM nicht simuliert werden kann. Dadurch können grobe Fehler entstehen. \cite[vgl.][Stevens \& Bony]{stevensbony}. Jedoch kann man aufgrund der Rechenkosten und Rechenzeit eine solche Herangehensweise mit parametrisierter Konvektion im simulieren großflächiger (globaler) Klimamodelle  nicht umgehen.\\
Will man nun jedoch die Auswirkungen des Klimawandel beziehungsweise der Klimamodell auf regionaler Ebene berechnen reichen die grob-skalierten GCM's nicht aus, da auf regionaler Ebene die Ortographie (wie z.B die Alpen oder die Pyrenäen), die flache (shallow) Konvektion, die zu Tiefenkonvektion wird und die lokalen Kältepools große Auswirkungen auf die lokalen Wettererscheinungen haben. Zudem hat die flache Konvektion durch die starke Rückwirkungen auf die Tiefenkonvektion auch eine Auswirkung auf globale Klimamodelle, wie es z.B. gut Anhand der Tropen \cite[vgl.][Teixeara et al.]{teixeracardoso} und der Madden-Julian Oszillation\cite[vgl.][Chen S. et al.]{chenshuyi} zu erkennen ist, wo aus flacher Konvektion Tiefenkonvektoin wird und daraus sich großflächige Wettererscheinungen bilden.\\
Aufgrund der Rückwirkung der konvektionserlaubenden regionalen Klimamodelle (CP-RCM)auf die GCM's stellt sich auch die Frage, ob denn gegenwärtige globale Klimamodelle bereit sind, sie als feed für RCM's zu verwenden.\\
Aber auch die Auflösung der gegenwärtigen Klimamodelle ist nicht immer ausreichend: z.B. können Turbulenzen und andere atmosphärische Mikroprozesse nicht von der derzeitigen Maschenweite der Simulationsgitter vollends erfasst und müssen um grobe Fehler zu vermeiden statistisch und dynamisch kombiniert downscaled werden. \cite[vgl.][Maraun et al.]{marauntowards}\\



