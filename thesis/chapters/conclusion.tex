Die einzelnen Modelldaten unterscheiden sich am stärksten durch die beiden unterschiedlichen globalen Klimamodelle (GCM's) mit denen sie betrieben wurden. Die Ergebnisse, welche durch das Betreiben mit den Re-Analysedaten erhalten wurden, waren viel näher an den Beobachtungsdaten als jene, welche durch das MPI-ESM-LR erhalten wurden. Was aber weiter nicht verwunderlich ist, da zweiteres ein in die Vergangenheit gerechnetes GCM ist und deshalb z.B. Extremwetterlagen nicht leicht bzw. nicht zeitlich korrekt abbilden kann.\vspace{1pt}\\

Die im Kapitel \ref{chap:mean} erhaltenen Ergebnissen deuten darauf hin, dass das regionale Klimamodell mit explizit simulierter Konvektion (CCLM5-0-9) im Mittel eine bessere Vorhersage gewährt, solange das GCM die tatsächlichen Gegebenheiten korrekt abbildet. Ansonsten scheint die Abweichung größer zu werden, vermutlich durch positive Rückkopplung der beiden regionalen Klimamodelle, da diese für die Simulation ineinander genested wurden: dies sieht in der Tablle \ref{tab:appendix} im Historical-Datensatz von ALP-3, wo sich durch den Betrieb mit dem GCM MPI-ESM-LR eine größere mittlere Abweichung (vgl. Abb.\ref{fig:mean_boxplots}) für ergibt.\\
Vergleicht man die Streuung der mittleren Abweichungen scheint die parametrisierte Konvektion weniger starke Ausreißer zu erzeugen, wie in der Abb.\ref{fig:mean_boxplots} und in Tabelle \ref{tab:appendix}zu erkennen ist. Daraus kann geschlossen werden, dass im Mittel die parametrisierte Konvektion ausreichend ist um ein GCM auf regionalen Maßstab herunterzubrechen.\\
Das downscaling in gebirgigen Gegenden scheint jedoch im RCM CCLM5-0-9 besser dargestellt werden, da in diesen Gebieten die parametrisierte Konvektion die größten Schwächen aufzeigt (vgl \cite{RCM}). Dies ist besonders gut in der Abbildung \ref{fig:mean_diff} zu erkennen, wo für den ALP-3 Datensatz am Alpenhauptkamm zwar starke aber kleinflächige Abweichungen zu erkennen sind. Die Abweichungen könnten wiederum durch das Nesting in das RCM CCLM4-8-17 oder auch ungenaue GCM - bzw. Re-Analysedaten entstanden sein. Das feintuning des Modells scheint im Mittel relativ gut zu sein. Jedoch müsste man das CCLM5-0-9 Modell in anderen Simulationen evaluieren um diese Aussage mit zu untermauern. \vspace{1pt}\\


Die Simulation von Starkniederschlägen(99. Quantil des Niederschlags), betrachtet über das ganze Jahr scheint das RCM mit explizit simulierter Konvektion am besten zu bewerkstelligen. Wieder ist dies stark davon abhängig, wie gut das betreibende GCM bzw. die Re-Analysedaten die Wirklichkeit abbildet. Dazu ist die Darstellung der Boxplots in Abb.\ref{fig:quantile_all_boxplots} ein maßgebliches Argument: Die Ausreißer scheinen zwar wieder für die parametrisierte Konvektion in einem engeren Rahmen zu liegen jedoch ist der mittlere Bias des 99. Quantils für die simulierte Konvektion betrieben mit den Re-Analysedaten näher bei 0.\\
Diese Abweichungen finden sich vor allem in gebirgigen Gebieten am Alpenhauptkamm und  Gebieten mit komplexen Wettersystemen wie z.B. Genua (siehe Abb.\ref{fig:quantile_alp3}). Dies lässt darauf schließen, dass für solche Gebiete die Modellkomponenten noch getuned werden müssen oder aber die GCM - bzw. Reanalysedaten zu grob sind um den Klimazustand in solch kleinen Gebieten korrekt abzubilden.\vspace{1pt}\\

Für die Starkregenereignisse (99. Quantil des Niederschlags) in den einzelnen Jahreszeiten scheint sich das Muster der Abweichung etwas anders abzuzeichnen als in den bisherigen Betrachtungen:\\
Im Frühling erlangt man durch die Modellierung mit parametrisierter Konvektion betrieben mit den Re-Analysedaten bessere Ergebnisse in Anbetracht ihrer geringe Streuung und dem geringen Bias. Für den Betrieb des RCM's mit dem GCM MPI-ESM-LR scheinen sich die Ausreißer durch die simulierte Konvektion in einem kleinerem Rahmen aufzuhalten. Trotzdem ist der Bias mit parametrisierter Konvektion (CCLM4-8-17) näher bei 0. Der Datensatz Evaluation ALP-3 sticht durch seine extremen Ausreißer hervor, was auf ein kleines Gebiet im Süden Kroatiens (am Rand des Abbildungsbereiches) zurückzuführen ist (vgl. Abb.\ref{fig:seasons_boxplots}). Beachtenswert ist, dass dieses Gebiet nur im Sommer keine so extremen Ausreißer produziert wie in allen anderen Jahreszeiten. Da das Gebiet am Rande des Gitters liegt, könnte es sich auch um eine Fehlabbildung im betreibenden Datensatz oder die in Kapitel \ref{chap:modells} besprochenen Randerscheinungen handeln. Betrachtet man kleinere Gebiete und deren Flächen-gemittelten Niederschlag, scheint sich für die Extremniederschläge im Frühling ein weitaus besseres bzw. Beobachtungs-näheres Muster abzubilden: vgl. Abb.\ref{fig:seasons:pr_over_undersim_eval_alp3} mit Abb.\ref{fig:seasons:hist_eur11:overundersim_mean}. Die Fluktuationen des Niederschlags scheint durch eine explizit simulierte Konvektion näher an der der Beobachtungsdaten zu liegen.\\
Im Sommer streuen sich die Abweichungen der Daten aus dem RCM mit simulierter Konvektion am wenigsten, da sich in dieser Jahreszeit die Konvektion am stärksten abzeichnet bzw. die stärksten Auswirkungen auf den Niederschlag haben, ist dies eine Bestätigung der Vorhersagequalität des CCLM5-0-9. Der Bias liegt zwar für den Historical-ALP3 Datensatz etwas über dem des EUR-11 Datensatzes aber wenn die Streuung mit berücksichtigt wird, ist für den Sommer die Vorhersage von Starkregenereignissen durch das RCM mit explizit simulierter Konvektion am besten gegeben. Dies fällt auch besonders gut auf, wenn die Fluktuation des Niederschlags gegen den mittleren Niederschlag aufgetragen wird: wieder ist diese, wie bereits im Frühling den Beobachtungsdaten näher als mit parametrisierter Konvektion. Betrachtet man die Fluktuation der Extremniederschläge mit parametrisierter Konvektion so scheinen diese in einer gegebenen Frequenz abzulaufen ohne direkten Bezug auf den Klimazustand bzw. den Beobachtungsdaten (vgl. Abb.\ref{fig:seasons:undersim_eval_eur11} mit Abb.\ref{fig:seasons:mean_alp3}).\\
Winter und Herbst wurden nicht gesondert betrachtet, doch in diesen Jahreszeiten zeichnet sich die Abweichung der Starkregenereignisse fast gleich ab wie im Frühling. Dies ist besonders gut in der Grafik \ref{fig:seasons_boxplots} zu erkennen. Beachtenswert dabei ist, dass der Bias des Datensatzes Historical EUR-11 im Herbst und des Datensatzes Evaluation EUR-11 im Winter das Minimum bildet. Des Weiteren muss beachtet werden, dass die Abweichungen im Winter besonders groß für das RCM mit simulierter Konvektion (CCLM5-0-9) ausfallen, da die Streuung und der Bias für beide Datensätze, Evaluation und Historical weit über dem Bias der Datensätze aus dem RCM mit parametrisierter Konvektion liegt.\\
\vspace{1pt}
Abschließend kann gesagt werden, dass das RCM mit simulierter Konvektion (CCLM5-0-9) über das gesamte Gebiet im Durchschnitt schlechter abschneidet als jenes mit parametrisierter Konvektion (CCLM4-8-17). Für kleinräumige gebirgige Gebiete liefert das CCLM5-0-9 bessere mittlere und auch starke Niederschlagsdaten als das CCLM4-8-17. Zudem sind Fluktuationen der Starkregenereignisse im Bezug zum mittleren Niederschlag näher an den Beobachtungsdaten.\\
Wie im Artikel \cite{RCM} Kapitel 3 angemerkt ist, wird besonders der Tagesgang durch die Modelle mit explizit modellierter Konvektion esser dargestellt. Dies ist mit den mir zur Verfügung stehenden Daten zwar nicht nachweisbar, jedoch zeichnet sich die bessere Darstellung auch im täglichen Gang des Niederschlags ab.\\