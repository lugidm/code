Die einzelnen Modelldaten unterscheiden sich am stärksten durch die beiden unterschiedlichen globalen Klimamodelle (GCM's) mit denen sie betrieben wurden. Die Ergebnisse, welche durch das Betreiben mit den Re-Analysedaten erhalten wurden, waren viel näher an den Beobachtungsdaten als jene, welche durch das MPI-ESM-LR erhalten wurden. Was aber weiter nicht verwunderlich ist, da zweiteres ein in die Vergangenheit gerechnetes GCM ist und deshalb z.B. Extremwetterlagen nicht leicht bzw. nicht zeitlich korrekt abbilden kann.\vspace{7pt}\\

Die im Kapitel \ref{chap:mean} erhaltenen Ergebnissen deuten darauf hin, dass das regionale Klimamodell im Mittel eine bessere Vorhersage gewähren, solange das GCM die tatsächlichen Gegebenheiten korrekt abbildet. Das RCM betrieben mit den Re-Analysedaten gibt mit dynamisch simulierter Konvektion (ALP-3) eine weitaus besseren \textbf{mittlere} Abweichung als das RCM mit parametrisierter Konvektion. Mit dem GCM MPI-ESM-LR betrieben ergibt sich jedoch eine leicht größere mittlere Abweichung des Biases (vgl. Abb.\ref{fig:yearly_mean_biases}) für den ALP-3 Datensatz.\\
Vergleicht man die Streuung der mittleren Abweichungen scheint durch die parametrisierte Konvektion weniger starke Ausreißer erzeugt zu werden, wie in der Abb.\ref{fig:mean_boxplots} zu erkennen ist. Daraus kann geschlossen werden, dass im Mittel die parametrisierte Konvektion ausreichend ist um ein GCM auf regionalen Maßstab herunterzubrechen.\vspace{7pt}\\

Wenn man die jährliche mittlere Abweichung eines Jahres betrachtet (Kapitel \ref{section:2002}) so ergeben sich leicht abweichende Aussagen über die unterschiedlichen Datensätze als für das Mittel eines Jahrzehnts: Die Streuung scheint zwar immer noch durch die dynamisch simulierte Konvektion stärker zu sein (vgl. Abb.\ref{fig:freq_2002}), Die Verteilungskurven der Häufigkeiten sind jedoch spitzer zulaufend über 0, was einer geringeren Abweichung im Mittel entspricht. Dies kann auch durch die in Tabelle \ref{tab:appendix} angegebenen mittleren Abweichungen nachvollzogen werden.\vspace{7pt}\\

Die Modellierung von Starkniederschläge, betrachtet über das ganze Jahr scheint die parametrisierte Konvektion am besten zu bewerkstelligen. Wieder ist dies stark davon abhängig, wie gut das GCM die Wirklichkeit abbildet, mit welchem das RCM betrieben wird. Dazu ist die Darstellung der Boxplots in Abb.\ref{fig:quantile_all_boxplots} ein maßgebliches Argument: Die Ausreißer scheinen zwar wieder für die parametrisierte Konvektion in einem engeren Rahmen zu liegen jedoch ist der mittlere Bias des 99. Quantils für die simulierte Konvektion betrieben mit den Re-Analysedaten um näher bei 0.\\
Diese Abweichungen finden sich vor allem in gebirgigen Gebieten am Alpenhauptkamm und  Gebieten mit komplexen Wettersystemen wie z.B. Genua (siehe Abb.\ref{fig:quantile_alp3}). Dies lässt darauf schließen, dass in solchen Gebieten die Simulation der Konvektion noch nicht ausgereift bzw. vollends durch das Modell erfasst wurde. Da besonders in diesen Regionen die simulierte Konvektion bessere Ergebnisse bringen sollte, wie D.Maraun et al. in \cite{maraun_value} postuliert.\vspace{7pt}\\

Für die Starkregenereignisse (99. Quantil des Niederschlags) in den einzelnen Jahreszeiten scheint sich das Muster der Abweichung etwas anders abzuzeichnen als in den bisherigen Betrachtungen:\\
Im Frühling dominiert die Modellierung mit parametrisierter Konvektion durch ihre geringe Streuung und den guten Bias, betrieben mit den Re-Analysedaten. Für die Ansteuerung des RCM's mit dem MPI-ESM-LR scheinen sich die Ausreißer durch die simulierte Konvektion besser abzubilden bzw. ist die Abweichung näher bei 0. Der Datensatz Evaluation ALP-3 sticht durch seine extremen Ausreißer hervor, was auf ein bestimmtes Gebiet im Süden Kroatiens zurückzuführen ist (vgl. Abb.\ref{fig:seasons_boxplots}). Beachtenswert ist, dass dieses Gebiet nur im Sommer keine so extremen Ausreißer produziert wie in allen anderen Jahreszeiten. Da das Gebiet am Rande des Gitters liegt, könnte es sich auch um eine Fehlabbildung im Datensatz handeln. Betrachtet man kleinere Gebiete und deren Flächen-gemittelten Niederschlag, scheint sich für die Extremniederschläge im Frühling ein weitaus besseres bzw. Beobachtungs-näheres Muster abzubilden: vgl. Abb.\ref{fig:seasons:pr_over_undersim_eval_alp3} mit Abb.\ref{fig:seasons:hist_eur11:overundersim_mean}. Die Niederschläge bezogen auf den Durchschnittsniederschlag kommen den beobachteten Daten am nächsten mit der simulierten Konvektion.\\
Im Sommer streuen sich die Abweichungen der Daten aus dem RCM mit simulierter Konvektion am wenigsten, da sich in dieser Jahreszeit die Konvektion am stärksten abzeichnet bzw. die stärksten Auswirkungen auf den Niederschlag haben, ist dies eine Bestätigung der Vorhersagequalität des RCM's. Der Bias liegt zwar für den Historical-ALP3 Datensatz etwas über dem des EUR-11 Datensatzes aber wenn die Streuung mit berücksichtigt wird, ist für den Sommer die Vorhersage von Starkregenereignissen durch das RCM mit dynamisch simulierter Konvektion am besten gegeben. Dies fällt auch besonders gut auf, wenn der Niederschlag gegen den mittleren Niederschlag einer Fläche aufgetragen wird. Das Muster des Niederschlags folgt um einiges besser den Beobachtungsdaten, wenn die Konvektion simuliert und nicht nur parametrisiert ist: Die Ausschläge von Niederschlagsextrema sind durch die parametrisierte Konvektion eine nahezu Vorhersagbare Frequenz, durch die simulierte Konvektion den Beobachtungsdaten treu (vgl. Abb.\ref{fig:seasons:undersim_eval_eur11} mit Abb.\ref{fig:seasons:mean_alp3}).
Winter und Herbst wurden nicht gesondert betrachtet, doch in diesen Jahreszeiten zeichnet sich die Abweichung der Starkregenereignisse fast gleich ab wie im Frühling. Dies ist besonders gut in der Grafik \ref{fig:seasons_boxplots} zu erkennen. Beachtenswert dabei ist, dass der Bias des Datensatzes Historical EUR-11 im Herbst und des Datensatzes Evaluation EUR-11 im Winter das Minimum bildet. Des Weiteren muss in Betracht gezogen werden, dass die Abweichungen im Winter besonders schlecht für das RCM mit simulierter Konvektion ausfällt, da die Streuung und der Bias für beide Datensätze, Evaluation und Historical weit über dem Bias der Datensätze aus dem RCM mit parametrisierter Konvektion liegt.\\
Abschließend kann gesagt werden, dass das RCM mit simulierter Konvektion im Schnitt schlechter abschneidet als jenes mit parametrisierter Konvektion aber die Frequenz der Niederschläge bzw. Extremausschläge besser dargestellt werden.