Die einzelnen Modelldaten unterscheiden sich am stärksten durch die beiden unterschiedlichen Globalen Klimamodelle (GCM's) mit denen sie betrieben wurden. Die Ergebnisse, welche durch das Betreiben mit den Re-Analysedaten erhalten wurden, waren viel näher an den Beobachtungsdaten als jene welche durch das MPI-ESM-LR erhalten wurden. Was aber weiter nicht verwunderlich ist, da zweiteres ein in die Vergangenheit gerechnetes GCM ist und deshalb z.B. Extremwetterlagen nicht leicht bzw. nicht zeitlich korrekt abbilden kann.\\
Die im Kapitel \ref{chap:mean} erhaltenen Ergebnissen deuten darauf hin, dass das regionale Klimamodell im Mittel eine bessere Vorhersage gewähren, solange das GCM die tatsächlichen Gegebenheiten korrekt abbildet. Das RCM betrieben mit den Re-Analysedaten gibt mit dynamisch simulierter Konvektion (ALP-3) eine weitaus besseren \textbf{mittlere} Abweichung als das RCM mit parametrisierter Konvektion. Mit dem GCM MPI-ESM-LR betrieben ergibt sich jedoch eine leicht größere mittlere Abweichung des Biases (vgl. Abb.\ref{fig:yearly_mean_biases}) für den ALP-3 Datensatz\\
Vergleicht man die Streuung der mittleren Abweichungen scheint durch die parametrisierte Konvektion weniger starke Ausreißer erzeugt zu werden, wie in der Abb.\ref{fig:mean_boxplots} zu erkennen ist. Dies 