Um ein Bild von der allgemeinen Übereinstimmung der simulierten Daten mit den tatsächlich beobachteten Daten zu erhalten wird in diesem Kapitel das Jahresmittel der simulierten und beobachteten Daten verglichen. Dadurch erhält man einen Überblick über die geographische Übereinstimmung der Regenzonen und Trockengebiete im abgebildeten Bereich.
\section{Herangehensweise}
\begin{enumerate}
	\item Berechnung des jährlichen Mittelwertes pro Gitterzelle über aller Jahre
	\item Subtraktion dieser Mittelwerte (Simuliert - Beobachtet): Mittelwerte pro Gitterzelle über diese Differenzen; ein Mittel über die gesamten zehn Jahre der Simulationsdaten	
\end{enumerate}
\section{Mittlerer Bias}
In diesem Unterkapitel soll das Verhalten der gemittelten Differenzen aller Datensätze aufgezeigt werden.
%\begin{figure}[h]
%	\includegraphics[width=\textwidth]{mean_n/yearly_mean_biases.jpg}
%	\caption{Die Biases der gemittelten Differenzen aller vier Datensätze aufgetragen gegen die Zeit}
%	\label{fig:yearly_mean_biases}
%\end{figure}
Wie man gut in den Abbildungen \ref{fig:mean_freq_plots} erkennen kann, liegt die Abweichung des Evaluation-Datensätze weit näher bei 0 als die des Historicals. Des Weiteren ist gut zu erkennen, dass die Abweichung des Datensatzes ALP-3 betrieben mit den Re-Analysedaten (Evaluation) die Beobachtungsdaten gut abbildet und somit im Mittel ein gutes Klimamodell für den Alpenraum darstellt. Beide Klimamodelle betrieben mit dem GCM MPI-ESM-LR liefern Ergebnisse, die eine starke Abweichung in positive Richtung aufweist, wobei hier kaum noch Unterschiede zwischen EUR-11 und ALP-3 auszumachen sind. Diese Abbildung spricht somit für die Qualität des regionalen Klimamodells, ALP-3 und aber auch gegen die Treffsicherheit des globalen Klimamodells.\\
In Abb.\ref{fig:mean_freq_plots} die Häufigkeiten der Abweichungen abgebildet, sowie ein Boxplot in Abb.\ref{fig:mean_boxplots} angelegt, wo zu erkennen ist, dass es einige große Ausreißer gibt. Auf diesen Fakt wird auch in Folge eingegangen, da es über das gesamte Gitter starke Schwankungen der Abweichung gab, welche sich vor allem an den Gebirgskämmen abzeichnete.\\

\begin{figure}[h!]
	\begin{subfigure}{0.49\textwidth}
		\includegraphics[width=\textwidth]{mean_n/frequenciesdif_mean_hist.jpg}
		\caption{Historical, EUR-11}
	\end{subfigure}
	\begin{subfigure}{0.49\textwidth}
		\includegraphics[width=\textwidth]{mean_n/frequenciesdif_mean_hist_apgd.jpg}
		\caption{Historical, ALP-3}
	\end{subfigure}
	\begin{subfigure}{0.49\textwidth}
		\includegraphics[width=\textwidth]{mean_n/frequenciesdif_mean_eval.jpg}
		\caption{Evaluation, EUR-11}
	\end{subfigure}
	\begin{subfigure}{0.49\textwidth}
		\includegraphics[width=\textwidth]{mean_n/frequenciesdif_mean_eval_apgd.jpg}
		\caption{Evaluation, ALP-3}
	\end{subfigure}
	\caption{Frequency-Plots der Differenzen des jährlichen Mittels des Niederschlags gemittelt über alle Jahre}
	\label{fig:mean_freq_plots}
\end{figure}
\begin{figure}[h!]
	\begin{subfigure}{0.49\textwidth}
		\includegraphics[width=\textwidth]{mean_n/eur_11_mean_historical_boxplot.jpg}
		\caption{Historical, EUR-11}
	\end{subfigure}
	\begin{subfigure}{0.49\textwidth}
		\includegraphics[width=\textwidth]{mean_n/alp3_mean_historical_boxplot.jpg}
		\caption{Historical, ALP-3}
	\end{subfigure}
	\begin{subfigure}{0.49\textwidth}
		\includegraphics[width=\textwidth]{mean_n/eur_11_mean_evaluation_boxplot.jpg}
		\caption{Evaluation, EUR-11}
	\end{subfigure}
	\begin{subfigure}{0.49\textwidth}
		\includegraphics[width=\textwidth]{mean_n/alp3_mean_evaluation_boxplot.jpg}
		\caption{Evaluation, ALP-3}
	\end{subfigure}
	\caption{Box-Plots der Differenzen des jährlichen Mittels des Niederschlags gemittelt über alle Jahre}
	\label{fig:mean_boxplots}
\end{figure}
\begin{figure}[h]
	\includegraphics[width=0.90\textwidth]{mean_n/dif_mean_hist_apgd.jpg}
    \caption{Differenzen der jährlichen Mittel, gemittelt über alle Jahre, Datensatz: Historical, ALP-3}
    \label{fig:alp3_dif_hist_mean}
\end{figure}
Um die geographische Verteilung der größten Abweichungen darzustellen wurden eine beispielhafte Darstellung gewählt, die die größten mittleren Abweichungen zeigt: der Datensatz Historical im regionalen Klimamodell ALP-3. Diese ist in Abbildung \ref{fig:alp3_dif_hist_mean} abgebildet.
Wie man gut in der Abbildung erkennen kann, ist im Allgemeinen über den orthographisch gediegenen Gegenden die Abweichung vom Beobachtungsdatensatz gering. Es ist gut ersichtlich, dass die Abweichungen in überwiegend gebirgigen Gegenden größer sind. Wie auch in den Abbildungen \ref{fig:mean_freq_plots} zu sehen ist, überwiegt eine Abweichung von $0.5-1$. Die Kurve ist auch leicht ins Rechte verschoben - dies bedeutet, dass mehr Niederschlage simuliert wurde als es tatsächlich gab. Dies könnte somit eine Schwäche des globalen Klimamodells darstellen, da sich dieses Muster auch in den anderen Datensätzen abzeichnet, wie im folgenden Kapitel erkenntlich gemacht werden soll.

\begin{comment}
\section{Beobachtungen eines Jahres: 2002} \label{section:2002}
Das Jahr 2002 wurde gewählt, da sich in diesem Jahr die stärksten Abweichungen zeigen. In diesem Kapitel soll es darum gehen, die örtliche Verteilung der Differenzen näher zu betrachten. Dazu wurden zunächst die Abweichungen des Jahresmittels bildlich für die beiden Evaluation-Datensätze dargestellt: Abb. \ref{fig:dif_mean_2002}. Wie man erkennen kann, häufen sich Analog zu den Differenzen, gemittelt über alle Jahre die Abweichungen besonders in den gebirgigen Gebieten, in den Ebenen scheint die Übereinstimmung mit den Beobachtungsdaten gut zu sein.\\
\begin{figure}[h]
		\begin{subfigure}{0.49\textwidth}
			\includegraphics[width=\textwidth]{mean_n/2002dif_mprs_hist-obs.jpg}
			\caption{Historical, EUR-11}
			\label{fig:dif_mean_2002:eur11_hist}
		\end{subfigure}
		\begin{subfigure}{0.49\textwidth}
			\includegraphics[width=\textwidth]{mean_n/2002dif_mprs_alp3hist-apgd.jpg}
			\caption{Historical, ALP-3}
			\label{fig:dif_mean_2002:alp3_hist}
		\end{subfigure}
		\begin{subfigure}{0.45\textwidth}
			\includegraphics[width=\textwidth]{mean_n/2002dif_mprs_eval-obs.jpg}
			\caption{Evaluation, EUR-11}
			\label{fig:dif_mean_2002:eur11_eval}
		\end{subfigure}
		\begin{subfigure}{0.45\textwidth}
			\includegraphics[width=\textwidth]{mean_n/2002dif_mprs_alp3eval-apgd.jpg}
			\caption{Evaluation, ALP-3}
			\label{fig:dif_mean_2002:alp3_eval}
		\end{subfigure}
	\caption{Differenzen des jährlichen Mittels über den Niederschlag im Jahr 2002}
	\label{fig:dif_mean_2002}
\end{figure}

Wie in den Grafiken bei Abb.\ref{fig:dif_mean_2002} zu erkennen ist bleibt in einem gewissen Bereich die Abweichungen über alle Datensätze am größten. Deshalb wurde dieser in Folge gesondert betrachtet. Der Bereich wurde in den Abbildungen gekennzeichnet.\\
Diese gekennzeichnete Fläche wurde aus allen Datensätzen ausgeschnitten und im Jahr 2002 über dessen Fläche gemittelt, somit ergab sich aus dem Rechteck ein Mittelwert für jeden Tag für jeden Datensatz. Von diesen Werten wurden dann die ebenfalls Flächen-gemittelten Beobachtungsdaten (APGD) abgezogen und im Diagramm in Abb.\ref{fig:diff_2002} gegen die Zeit aufgetragen.\\
\begin{figure}[h]
	\includegraphics[width=0.95\textwidth]{mean_n/differences_2002.jpg}
	\caption{Die Differenzen für den in der Abb. \ref{fig:dif_mean_2002} gekennzeichneten Bereich im Jahr 2002. Es muss beachtet werden, dass hier die tägliche Inkonsistenz betrachtet wird, dies ist nicht ein Fehler des Modells.}
	\label{fig:diff_2002}
\end{figure}\\
Man sieht in Abb.\ref{fig:diff_2002}, dass die Abweichungen über das Jahr verteilt stark fluktuieren. Beachtenswert ist dabei, dass die Ausschläge der Kurven für den ALP-3 Datensatz deutlich größer und auffallend in das Positive verschoben sind. Die Kurvenform der beiden, Historical und Evaluation Datensätzen sind nahezu identisch, was auf Fehler in den Antriebsdaten zurückzuführen ist. Da die Ausschläge gemittelt über das ganze Gebiet (vgl. Abb.\ref{fig:mean_boxplots} und Abb.\ref{fig:mean_freq_plots}) für den ALP-3 Datensatz deutlich besser sind ist es in diesem gesondert betrachteten Gebiet wahrscheinlich zu einem Overfitting des Modells gekommen und die extremen Ausschläge sind Fluktuationen die daraus resultieren. Um die Ergebnisse der beiden RCM's EUR-11 und ALP-3 besser untersuchen zu können wurden die zwei Datensätze mit den größten Abweichungen (Historical aus EUR-11 und ALP-3) in Abb.\ref{fig:precip_2002} dargestellt.\\

\begin{figure}[b]
	\includegraphics[width=0.90\textwidth]{mean_n/historical_2002.jpg}
	\caption{Niederschlag im Jahr 2002 für den in Abb.\ref{fig:dif_mean_2002} gekennzeichneten Bereich aus den Datensätzen Historical von ALP-3 bzw. EUR-11 gegen die Beobachtungsdaten aus APGD}
	\label{fig:precip_2002}
\end{figure}
\begin{figure}[b]
	\begin{subfigure}{0.49\textwidth}
		\includegraphics[width=\textwidth]{mean_n/2002frequenciesdif_mprs_hist-obs.jpg}
		\caption{EUR-11, Historical}
	\end{subfigure}
	\begin{subfigure}{0.49\textwidth}
		\includegraphics[width=\textwidth]{mean_n/2002frequenciesdif_mprs_alp3hist-apgd.jpg}
		\caption{ALP-3, Historical}
	\end{subfigure}
	\begin{subfigure}{0.49\textwidth}
		\includegraphics[width=\textwidth]{mean_n/2002frequenciesdif_mprs_eval-obs.jpg}
		\caption{EUR-11,Evaluation}
	\end{subfigure}
	\begin{subfigure}{0.49\textwidth}
		\includegraphics[width=\textwidth]{mean_n/2002frequenciesdif_mprs_alp3eval-apgd.jpg}
		\caption{ALP-3, Evaluation}
	\end{subfigure}
	\caption{Häufigkeit gewisser Abweichungen für das gemittelte Jahr 2002 in den beiden Historical - Datensätzen (betrachtetes Gebiet: \underline{gesamter} Alpenraum)}
	\label{fig:freq_2002}
\end{figure}

In der Abb. \ref{fig:precip_2002} liegen die im regionalen Klimamodell (ALP-3) die vorhergesagten Niederschlagsmengen für größere Niederschlagsmengen z.B. im Mai 2002 um einiges näher an den beobachteten Daten. Es werden jedoch auch manche Niederschlagsereignisse schlichtweg überschätzt, was sich dann im Gesamtbild schlecht auswirkt (vgl.\ref{fig:freq_2002}: die Kurve ist deutlich ins Positive verschoben.\\
Das Klimamodell welches die Konvektion nicht simuliert (EUR-11) scheint dauernd eine gewisse Fluktuation im Niederschlag zu errechnen, ohne gezielt Starkregenereignisse zu reproduzieren. Dies ist gut im Bereich April 2002 in der Abb. \ref{fig:precip_2002} zu erkennen: Die Fluktuationen scheinen gewissermaßen ständig aufzutreten. Das ist auf die Parametrisierung der Konvektion zurückzuführen. Da dadurch auch keine längeren Regenpausen vorkommen können, ist dies ein großes Manko in der Vorhersagekraft von diesen Modellen für die zukünftige Entwicklung des Klimas auf regionaler Ebene (siehe ''Bias Correction, Quantile Mapping, and Downscaling: Revisiting the Inflation Issue'' von D.Maraun \cite{biasMaraun}). \\
Das Konvektion-simulierende Klimamodell ALP-3 scheint besonders im Winter und im Herbst Niederschlagsextrema zu überschätzen. Im Sommer  bzw. Frühling stimmen die Daten relativ gut überein. Manche Extrema sind zwar zeitlich verschoben, jedoch muss bemerkt werden, dass die historischen Tage nichts mit den simulierten Tage im Klimamodell gemein haben, somit darf die zeitliche Verteilung nicht zu genau genommen werden. Sondern es muss eine gewisse Toleranz, wann ein Starkregenereignis eintritt mit-einberechnet werden.\\
\end{comment}
\section{Zusammenfassung}\label{sec:zusammenfassung_01}
Wie in diesem Kapitel gezeigt wurde, liegen die Mittelwerte beider regionalen Klimamodell ähnlich verteilt vor. Da dadurch die Extremniederschläge aus den Daten ausgemittelt werden und auch der Mittelwert des Niederschlags nicht das Klima einer Region beschreibt, kann der Mittelwert nicht zur vollständigen Evaluation eines Modells herangezogen werden. Nichtsdestotrotz gibt der Mittelwert einen guten Aufschluss darüber, wo die größten Abweichungen herrschen (siehe Abbildung \ref{fig:alp3_dif_hist_mean}). Jedoch sobald man ein Jahr oder auch die gesamte Zeitspanne genauer betrachten will steht man vor dem Problem, dass die simulierten Tage nicht direkt mit den historischen Tagen in Verbindung gebracht werden dürfen. Um nun diese tägliche Inkonsistenz aus der Betrachtung Außenvorzulassen soll im nächsten Kapitel auf die alljährlichen Extremwettererscheinungen eingegangen werden, wodurch diese Inkonsistenz in den Kalendertagen umgangen wird.